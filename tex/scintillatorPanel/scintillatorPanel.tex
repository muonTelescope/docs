\section{Scintillator Panel}
Large scintillator panel with embedded fibre loop to channel light to attached photo-sensor. The tile has a loop pattern with a single fibre to attempt to gather light from as much of the effective area of scintillator and minimize losses from tight curves and open fibre ends. The fibre groove follows along the top surface and comes out perpendicular to end face. A rendering of a assembled panel is shown in Figure~\ref{fig:assembledPanel}.

\fig{./tex/scintillatorPanel/img/scintillatorPanel}{Fully assembled Scintillator panel.}{assembledPanel}

\subsection{Theory}
The scintillator panel is a doped plastic (available from many suppliers \href{http://www.eljentechnology.com/}{Eljen} being the one used in our assemblies) that emits light in response to being struck by an electron, alpha particle, ion, or high energy photon. The light emitted (blue in our case) is then expected to hit the embedded fibre that can absorb that light, and retransmit at the lower green frequency based on the characteristics of the fibre. This green light is captured and guided by the fibre to the end of finger where it can be read out. Each event is expected to produce about 30 photons, with a lower number reaching the sensor placed on the fibre side. The outer surface has coatings applied to shield from external light and attempt to maximise efficiency by reflecting internal light. A exploded view of all the components is shown in Figure~\ref{fig:explodedPanel}.

\subsection{Assembly}
Currently the parts must be hand assembled although further production may yield an injection moulding scheme. The following lists the procedure to construct a single plate.

\subsubsection{Milling}
The fingers are cut from 10mm thick scintillator and are $\SI{145}{\milli\meter}$ x $\SI{145}{\milli\meter}$. The groove is cut with a ball mill under water flow to prevent deformation of the plastic due to heat. The milled sides are then polished (Novus plastic polish works well) and then are washed with alcohol.

\subsubsection{Gluing}
The optical fibres are cut to length (score and breaking) and the end is polished. It is placed into the groove with a two-part optical cement. This binds the fibre to the milled plastic providing an optical bond. Forcing the fibre to match the contour milled.

\fig{./tex/scintillatorPanel/img/scintillatorPaneExploded}{Exploded view showing coatings and fibre.}{explodedPanel}

\subsubsection{Covering}
The fingers are then covered with either degreased aluminium foil and electrical tape or a coating of reflective spray followed by a dip inside rubberized coating (successful tests have been done with \href{http://www.hobbyrecreationproducts.com/collections/spazstix-ultimate-mirror-chrome}{Spaz Stix Ultimate Mirror Chrome} and \href{https://plastidip.com/}{Plasti Dip}) providing a more even surface finish and reduced labour. This covering shields external light, and provides higher efficiency by capturing stray photons by reflecting them back towards the fibre optic. 

\pagebreak
